\documentclass{article}
\usepackage[utf8]{inputenc}

\title{Probability and Statistics notes}
\author{Aditya Mehrotra}
\date{April 2021}

\begin{document}

\maketitle

\tableofcontents

\newpage

\section{Probability}

\subsection{Basic mathematical definitions}

\textbf{Compliment:} Given a set of of elements $A$ where $A \subseteq S$ for some set of elements $S$, we define the compliment of $A$ in $S$ as the set of elements of $S$ that are not in $A$. Mathematically, this is expressed as $A^c$ or $S - A$.

\vspace{2 mm}

\noindent
\textbf{Difference:} The difference between sets $A$ and $B$ is all elements that are in $A$ but not in $B$. Mathematically, this is expressed as $A - B$

\vspace{2 mm}

\noindent
\textbf{Disjoint:} $A$ and $B$ are disjoint if they have no common elements. Mathematically, this is when $A \cap B = \emptyset$

\vspace{2 mm}

\noindent
\textbf{Product of sets:} The cartesian product of two sets $S$ and $T$, denoted by $S \times T$:
$$S \times T = \{(s, t) \mid s \in S, t \in T\}$$

\vspace{2 mm}

\noindent
\textbf{Cardinality of a set:} The number of elements in a set $S$, mathematically denoted by $|S|$

\vspace{2 mm}

\noindent
\textbf{Inclusion exclusion principle:}
$$|A \cup B| = |A| + |B| - |A \cap B|$$


\subsection{Basic probability terminology}

\textbf{Experiment:} A repeatable procedure with well-defined possible outcomes 

\vspace{2 mm}

\noindent
\textbf{Sample space:} The set of all possible outcomes. This is denoted by $\Omega$ and sometimes by $S$.

\vspace{2 mm}

\noindent
\textbf{Event:} A subset of the sample space

\vspace{2 mm}

\noindent
\textbf{Probability function:} A function assigning a probability for each outcome. More specifically, for a discrete sample space $S$, a probability function $P$ assigns each outcome $\omega \in S$ a number $P(\omega)$. This number is called the probability of $\omega$.

\vspace{1 mm}

\noindent
P must satisfy two rules:
\begin{enumerate}
    \item[] Rule 1: $0 \leq P(\omega) \leq 1$ (probabilities are between 0 and 1)
    \item[] Rule 2: Given $S = \{\omega_1, \omega_2, \omega_3, ..., \omega_n\}$, $\displaystyle\sum_{j = 1}^{n} P(\omega_j) = 1$. 
    \\
    (The sum of probabilities of all possible outcomes add to 1)
\end{enumerate}

\noindent
Additionally, the probability of an event $E \subseteq S$ is the sum of the probabilities of all outcomes in $E$. More specifically,
\[P(E) = \sum_{\omega \in E} P(\omega)\]

\vspace{2 mm}

\noindent
\textbf{Discrete sample space:} A listable sample space, can be finite or infinite.

\subsection{Basic probability rules}
For some events $A$, $L$ and $R$ contained in a sample space $\Omega$:

\noindent
\begin{enumerate}
    \item[] $P(A^c) = 1 - P(A)$
    \item[] If $L$ and $R$ are disjoint, then $P(L \cup R) = P(L) + P(R)$
    \item[] If $L$ and $R$ are not disjoint, we have the inclusion-exclusion principle:
    \[P(L \cup R) = P(L) + P(R) - P(L \cap R)\]
\end{enumerate}


\subsection{Conditional Probability}

\textbf{Conditional probability:} The probability of $A$ given $B$. Mathematically, this is expressed as $P(A|B)$. More rigorously, given events $A$ and $B$, the conditional probability of $A$ given $B$ is,

$$ P(A|B) = \frac{P(A \cap B)}{P(B)} \textit{, provided that } P(B) \neq 0$$

\noindent
This gives us the multiplication rule,
$$P(A \cap B) = P(A|B) \cdot P(B)$$


\noindent
\textbf{Law of Total Probability:} Suppose the sample space $\Omega$ is divided into $n$ disjoint events, $B_1, B_2, ..., B_n$. Then for any event $A$:
$$P(A) = P(A \cap B_1) + P(A \cap B_2) + ... + P(A \cap B_n) $$
$$P(A) = P(A|B_1)P(B_1) + P(A|B_2)P(B_2) + ... + P(A|B_n)P(B_n) $$


\subsection{Independence}
\end{document}
